%%==================================================
%% abstract.tex for BIT Master Thesis
%% modified by 朱杰
%%==================================================

\begin{abstract}
随着科技的进步和互联网的不断发展,在线社交网络如今已经成为互联网时代人们沟通与协作的重要载体。特别是随着微信、微博、Facebook和Twitter等国内外的社交类平台软件的出现,使得人们可以更加有高效的沟通交流,人们的社交重心逐渐由线下更多的转到了线上。在这样的背景下,如果将社交平台中所有用户抽象成点,而用户与用户之间的关联抽象成边,就抽象出了一张网络关系图。在网络内部连接比较紧密的节点子集合对应的子网络被称之为社区。网络图中包含一个个社区的现象称之为社区结构。社区结构是社交网络结构的一个普遍特征。而给定一个网络图,找出其社区结构的过程就叫做社区发现。对社交网络的社区发现在商业个性化推荐、舆情控制以及社交网络数据的分布式存储等领域均有很大的意义。

在查阅了大量社交网络、社区发现、图的划分以及图的聚类等相关资料之后,考虑到社交网络大规模以及“小世界”、“无标度”的特征,决定在简单快速的标签传播算法(LPA)的基础上进行改进,解决标签传播算法不稳定的问题。同样在重叠社区发现上,对多标签传播算法(COPRA)进行改进。下面是本文的主要工作:

(1)在非重叠社区发现上,提出了一种稳定的基于标签传播的非重叠社区发现算法,简称CDABSLP。该算法在标签更新的时候采用异步更新的策略;然后提出了一种综合节点自身重要性以及邻居节点重要性的节点影响力模型,将节点更新顺序设为节点影响力的降序;最后在标签选择方式上,提出了一种标签影响力模型。通过验证实验证明CDABSLP算法在具有良好的稳定性的同时兼具较高的运行效率,在多个评价指标之上社区发现准确率均有不错的表现。

(2)在重叠社区发现上,提出了一种稳定的基于标签传播的重叠社区发现算法,简称OCDABSLP。在简单分析了原始COPRA算法存在的缺陷之后,针对相应的缺陷介绍了稳定性解决方案。OCDABSLP算法采用了同步更新标签的策略,这点与CDABSLP算法不同;在标签选择方式上,对于邻居节点标签隶属度均小于既定阈值,且含有多个不同的最大隶属度标签的时候,选择最大隶属度标签中标签影响力模型计算值最大的那一个进行更新。在算法最终收敛之后,根据每个节点含有的标签将其划分到相应的社区,含有多个标签的节点即对应着存在于多个社区的重叠节点。通过真实网络以及人工网络中的验证实验,可知CDABSLP算法在具有良好的稳定性的同时兼具较高的运行效率,在多个评价指标之上社区发现准确率均有不错的表现。


% 本文……。({\color{blue}{摘要是一篇具有独立性和完整性的短文,应概括而扼要地反映出本论文的主要内容。包括研究目的、研究方法、研究结果和结论等,特别要突出研究结果和结论。中文摘要力求语言精炼准确,硕士学位论文摘要建议500$\sim$800字,博士学位论文建议1000$\sim$1200字。摘要中不可出现参考文献、图、表、化学结构式、非公知公用的符号和术语。英文摘要与中文摘要的内容应一致。}})

\keywords{社交网络; 社区发现; 标签传播;节点影响力}
\end{abstract}

\begin{englishabstract}

In recent years, with the development of science and technology and wite the continued popularity of the Internet, online social network has now become the most basic part of our life. The emergence of social network software such as WeChat and Weibo at home and abroad has enabled people to communicate more effectively. The social focus of people has gradually shifted from offline to online. In this context, if all users in the social network platform are abstracted into nodes, and the association between users and users is abstracted into edges, a network diagram is abstracted. The sub-network corresponding to the sub-collection of nodes that are more closely connected within the network is called a community. The phenomenon of including a community in the network map is called a community structure. Community structure is a common feature of social network structure. Given a network map, the process of finding its community structure is called community detection. The discovery of the social network community has great significance in the areas of commercial personalized recommendation, public opinion control, and distributed storage of social networking data.

After reviewing a large number of related data such as social networks, community detection, map partitioning, and graph clustering, we decided to use simple and rapid label propagation in light of the large scale of social networks and the characteristics of “small-world” and “scale-free”. The algorithm (LPA) is improved on the basis of solving the problem of unstable label propagation algorithm. Also in the overlap community detection, the multiple label propagation algorithm (COPRA) is improved. Here are the main tasks of this article:

(1) In order to obtain stable non-overlapping community detection results, a non-overlapping community detection algorithm based on stable label propagation is proposed, abbreviated as CDABSLP. The algorithm adopts the strategy of asynchronous updating when the tag is updated; then proposes a node influence model that integrates the importance of the node itself and the importance of the neighbor nodes, sets the node update order as the descending order of the node influence, and finally selects the tag. In terms of methods, a label influence model was proposed. The verification experiment proves that the CDABSLP algorithm has both good stability and high operating efficiency, and the community has found a good performance in the evaluation of the above indicators.

(2) In order to obtain stable non-overlapping community detection results, a non-overlapping community detection algorithm based on stable label propagation is proposed, referred to as OCDABSLP. The algorithm adopts the strategy of synchronous updating. In the process of tag updating, the influence of the node is introduced into the calculation formula of the membership degree of the tag. The intensity of the influence of these tags is obtained, and the tag with the greatest influence is retained. The traditional COPRA algorithm replaces one of the tags randomly. Method to improve the stability of the algorithm. The verification experiment proves that the OCDABSLP algorithm solves the problem of COPRA algorithm instability, and can detect the optimal overlapping community structure. It shows the applicability of the stability strategy proposed in the CDABSLP algorithm in overlapping community detection algorithm COPRA algorithm.
   
\englishkeywords{Social network; community detection; label propagation;  node influence}

\end{englishabstract}
