%%==================================================
%% abstract.tex for BIT Master Thesis
%% modified by 朱杰
%%==================================================

\begin{abstract}
随着科技的进步和互联网的不断发展,特别是微信、微博等国内外社交软件的出现,使得人们可以更加高效的沟通交流,人们的社交重心逐渐由线下转到了线上。在线社交网络如今已经成为互联网时代人们沟通与协作的重要载体。对社交网络的社区发现研究广泛应用于商业个性化推荐、舆情控制以及社交网络数据的分布式存储等领域。标签传播算法(Label Propagation Algorithm,简称LPA)由于简单、易理解、快速且无需任何先验信息得到广泛沿用,然而应用LPA算法最终得到的社区划分结果具有随机性,因此算法并不稳定。
% 随着科技的进步和互联网的不断发展,特别是微信、微博等国内外社交软件的出现,使得人们可以更加高效的沟通交流,人们的社交重心逐渐由线下转到了线上。在线社交网络如今已经成为互联网时代人们沟通与协作的重要载体。对社交网络的社区发现研究在商业个性化推荐、舆情控制以及社交网络数据的分布式存储等领域均有很大的意义。虽然目前已经有不少社区发现算法被提出,但是单独针对社交网络的算法并不多,且这些算法依然存在着一些问题,例如需要社区的先验信息、时间复杂度偏高、无法识别重叠社区等。标签传播算法(Label Propagation Algorithm,简称LPA)自被提出以来就备受学者们的关注,它简单、易理解、快速且无需任何先验信息,然而LPA算法最终得到的社区划分结果可能具有随机性,因此算法并不稳定。

针对上述问题,本文分析了大量社交网络、社区发现、图划分以及图聚类等相关资料,最终以标签传播思想为基础,提出了稳定的非重叠和重叠社区发现算法,具体算法如下:
% 在查阅了大量社交网络、社区发现、图的划分以及图的聚类等相关资料之后,考虑到社交网络大规模以及“小世界”、“无标度”的特征,决定在简单快速的标签传播算法(LPA)的基础上进行改进,解决标签传播算法不稳定的问题。同样在重叠社区发现上,对多标签传播算法(COPRA)进行改进。下面是本文的主要工作:

(1)针对标签传播算法的不稳定问题,提出了一种基于稳定标签传播的非重叠社区发现算法(Community Detection Algorithm Based on Stable Label Propagation,简称CDABSLP)。该算法在标签更新时采用异步更新策略;然后提出了一种综合节点自身重要性以及邻居节点重要性的节点影响力模型,按照节点影响力大小降序排列的顺序更新标签;在标签选择方式上,在节点影响力的基础上提出了标签影响力模型,选择标签影响力最大的标签对原始标签进行更新;最终根据每个节点对应的标签进行社区划分。实验结果表明,CDABSLP算法具有较好的稳定性和较高的运行效率,且能够获得较好的社区划分质量。
%(1)在非重叠社区发现上,提出了一种稳定的基于标签传播的非重叠社区发现算法,简称CDABSLP。该算法在标签更新的时候采用异步更新的策略;然后提出了一种综合节点自身重要性以及邻居节点重要性的节点影响力模型,将节点更新顺序设为节点影响力的降序;最后在标签选择方式上,提出了一种标签影响力模型。通过验证实验证明CDABSLP算法在具有良好的稳定性的同时兼具较高的运行效率,在多个评价指标之上社区发现准确率均有不错的表现。

(2)针对多标签传播算法(Community Overlap Propagation Algorithm,简称COPRA)的不稳定问题,提出了一种基于稳定标签传播的重叠社区发现算法(Overlapping Community Detection Algorithm Based on Stable Label Propagation,简称OCDABSLP)。该算法在标签更新时采用同步更新策略;在标签选择方式上,对于邻居节点标签隶属度均小于既定阈值,且含有多个不同的最大隶属度标签的情况,选择其中标签影响力最大的进行更新;最终根据每个节点对应的标签进行社区划分。实验结果表明,OCDABSLP算法具有较好的稳定性和较高的运行效率,且能够对重叠社区进行较好的识别。
% (2)在重叠社区发现上,提出了一种稳定的基于标签传播的重叠社区发现算法,简称OCDABSLP。在简单分析了原始COPRA算法存在的缺陷之后,针对相应的缺陷介绍了稳定性解决方案。OCDABSLP算法采用了同步更新标签的策略,这点与CDABSLP算法不同;在标签选择方式上,对于邻居节点标签隶属度均小于既定阈值,且含有多个不同的最大隶属度标签的时候,选择最大隶属度标签中标签影响力模型计算值最大的那一个进行更新。在算法最终收敛之后,根据每个节点含有的标签将其划分到相应的社区,含有多个标签的节点即对应着存在于多个社区的重叠节点。通过真实网络以及人工网络中的验证实验,可知CDABSLP算法在具有良好的稳定性的同时兼具较高的运行效率,在多个评价指标之上社区发现准确率均有不错的表现。

% 本文……。({\color{blue}{摘要是一篇具有独立性和完整性的短文,应概括而扼要地反映出本论文的主要内容。包括研究目的、研究方法、研究结果和结论等,特别要突出研究结果和结论。中文摘要力求语言精炼准确,硕士学位论文摘要建议500$\sim$800字,博士学位论文建议1000$\sim$1200字。摘要中不可出现参考文献、图、表、化学结构式、非公知公用的符号和术语。英文摘要与中文摘要的内容应一致。}})

\keywords{社交网络; 社区发现; 标签传播;节点影响力}
\end{abstract}

\begin{englishabstract}

With the advancement of science and technology and the continuous development of the Internet, especially the appearance of social applications such as WeChat and Weibo, people can have more efficient communication and cooperation. People's social focus has gradually shifted from offline to online. Online social network has now become an important carrier for people to communicate and collaborate. Community detection research on social network has great significance in the areas of commercial personalized recommendation, public opinion control, and distributed storage of data of social network. Although many community detection algorithms already have been proposed, only a few algorithms focus on social network. Besides, these algorithms still have some problems, such as the need for prior information of community, high time complexity, and inability to identify overlapping communities. The Label Propagation Algorithm (LPA) has attracted the attention of scholars since its appearance. It is simple, easy to understand, fast, and does not require any prior information. However, the community detection results obtained by the LPA algorithm may be random, so the algorithm is not stable.

In view of the above problems, this paper analyzes a large number of related data such as social network, community detection, graph partitioning, and graph clustering. Based on the idea of label propagation , a non-overlapping and  an overlapping stable community detection algorithm are proposed. The specific algorithms are as follows:

(1) Aiming at the instability problem of label propagation algorithm, a non-overlapping Community Detection Algorithm Based on Stable Label Propagation (CDABSLP) is proposed. The algorithm adopts the asynchronous updating strategy when the label is updated. Then a Node Influence Model that integrates the importance of the node itself and the importance of the neighbor nodes is introduced. The labels are updated in the descending order of the node influence. In the way of label selection, the Label Influence Model based on the Node Influence Model was proposed. The label with the biggest label influence was selected to update the original label. Finally the community was divided according to the corresponding label of each node. The experimental results show that the CDABSLP algorithm has good stability, high operating efficiency and good community division quality.
   
(2) Aiming at the instability problem of Community Overlap Propagation Algorithm (COPRA), an Overlapping Community Detection Algorithm Based on Stable Label Propagation (OCDABSLP) is proposed. The algorithm adopts the synchronous updating strategy when the label is updated. In the way of label selection, when the degree of membership of the neighboring node label is less than the predetermined threshold, and there are multiple different maximum degree of membership labels, the label with the biggest influence is selected to update the original label.  Finally, the community is divided according to the corresponding label of each node. A node with multiple labels corresponds to an overlapping node that exists in multiple communities. Experimental results show that the OCDABSLP algorithm has good stability, high operating efficiency and good overlapping community division quality.

\englishkeywords{Social network; Community detection; Label propagation;  Node influence}

\end{englishabstract}
