%%==================================================
%% chapter01.tex for BIT Master Thesis
%% modified by 朱杰
%%==================================================
\chapter{绪论}
%\label{chap:intro}
\section{研究背景和意义}

近年来,随着科技的发展和网络的不断普及,在线社交网络如今已经成为互联网时代最为基础的一部分。诸如微信、微博、Facebook、Twitter和GitHub等等国内外的社交类平台软件的出现,使得人们可以更加有高效的沟通交流。在如今的移动互联网时代下,人们的社交重心由线下更多的转到了线上。线上的社交也确实带来了很多的便捷。人们不再有地域的限制,可以轻松的与亲朋好友时刻保持联系;人们通过社交平台可以迅速的认识了解一个人并与之成为朋友;人们可以扮演起在日常生活中无法扮演的角色,任何人都可以成为信息的分享者和传播者。

如今这些成熟的社交软件每天都会产生海量的用户数据。在这个大数据的时代,这些看似杂乱无章、毫无交集的数据中,其实蕴含了丰富的信息等待着人们去挖掘与分析。在这样的背景下,如果将社交平台中所有用户抽象成点,而用户与用户之间的关联抽象成边,这就抽象出了一张网络关系图,那么对于在互联网这个虚拟世界中形成的这样一张巨大而又复杂的社交网络的研究与分析就显得意义非凡。

正所谓物以类聚,人以群分。社交网络中亦是如此。网络图内部连接比较紧密的节点子集合对应的子图称之为社区。网络图中包含一个个社区的现象称之为社区结构。社区结构是网络的一个普遍特征。而给定一个网络图,找出其社区结构的过程就叫做社区发现(community detection)。挖掘社交网络中的社区在人物分析、商业个性化推荐和舆情控制等领域有着很关键的作用。单单以商业个性化推荐而言,为了获得更大的用户群体以获取更多的流量关注,或者刺激促进用户更多的消费,在线求职招聘类平台要为用户推荐适合用户需求的职位,在线购物消费类平台要为用户推荐符合用户需求的商品,在线社交网络平台要为用户推荐和用户兴趣相投或者相关的好友,在线新闻媒体平台要为用户推荐符合用户口味的相关讯息。在这样的背景下,一个优秀的个性化推荐系统就显得尤为重要。在推荐系统中,免不了要对用户群体进行分类,而社区发现所形成的社区在此就有着先天的优势。直白的说,对社交网络的社区发现,无非也就是对社交网络中的用户分类(或者说聚类)。在同一个社区内的用户群体,往往有着相似的兴趣爱好。因此如果对整个用户群体进行社区发现,那么比如说在为用户进行商品个性化推荐时,可以重点关注与该用户在同一个社区的其他用户购买过而他未购买过的商品,这样用户的接受率也会高很多。当然,推荐系统是一个复杂的AI系统,利用了远不止社区发现这一种手段。

对社区发现算法的研究其实远不止社交网络这一领域,社区发现其实是对复杂网络的一种分析手段。除了社交网络,还有科学文献引用网络、生物学分子结构分析等等领域也都使用着社区发现算法。截至当前,多年来的众多学者的贡献使得对社区发现算法的研究已经是比较成熟了,但是依然存在着问题。现如今的社交网络数据往往是海量的,对如此大规模数据的社交网络进行社区发现时,过往的一些经典算法显然已经无法胜任。因此,在这样的背景下,提出一种适应于当前大规模社交网络的社区发现算法就显得尤为迫切了。

\section{国内外研究现状}
%\label{sec:***} 可标注label

本文这里将现有的社区发现算法分为三大类:基于链接的方法,基于内容的方法和融合链接和内容的方法。基于链接的方法也就是基于网络拓扑结构的方法。它将社交网络看做一张网络图,用户为节点,关系为边。基于内容的方法主要是基于社交网络中用户的个人信息和发表过的内容来进行社区发现。基于内容的方法也可被称为基于主题的方法。而融合的方法则同时关注网络拓扑结构和用户信息,以此来获取更高质量的社区。

\subsection{基于链接的社区发现算法}
%\label{sec:features}

物以类聚人以群分。人们总是习惯于和自己相似的人结交关系。不论是友谊还是工作关系,任何关系网络中人们的缔结方式均有此趋势。因此,在社交网络中两个用户之间所谓的链接关系被认为是一个可以证明彼此之间是有着某些共同点的证据。这也有利于发现社区。这个发现最早是被McPherson等人\cite{Mcpherson2001Birds}提出的,文中提到了同质性原则“相似性产生联系”。这也被认为是大部分社区定义的最早的参考准则。

在基于链接的社区发现算法中,社交网络是由一张图为模型,节点代表社区成员,边代表成员之间的关系或者交互。这里社区所需的凝聚力属性就是成员之间的链接。在社区之中链接是较为密集的,而在社区之间链接相对而言较为稀疏。分别将原始图结构中的组件和派系当做是已知的社区\cite{Fortunato2009Community}。然而,更多的更有意义的社区可以通过基于图划分(聚类)的方法来检测得到,其目标就是尽可能减少社区之间边的数量。这样一来,在一个社区中的节点之间就可以有更多的内连接,而与别的社区中节点的外连接就可以减少。大部分的方法都是基于二分迭代:不断地将一个社团划分为两个社团。然而在复杂网络中社区的数量显然是无法预先得知的。在这个层面上,Girvan-Newman的算法是最为广泛使用的基于链接的社区发现算法\cite{2002Community}。GN算法的基本思想是删除那些社区之间的连接,那么剩下的每个连通部分就是一个社区。那么问题来了,哪些才是社区之间的边呢?作者巧妙地借助了最短路径解决了这个问题。GN算法中定义一条边的边介数(betweeness)为网络中所有节点之间的最短路径中通过这条边的数量,而边介数高的边要比介数低的边更可能是社区之间的边。其实,这也比较好理解,因为两个社区中的节点之间的最短路径都要经过那些社区之间的边,所以它们的边介数会很高。在社区发现中几乎不可能预先知道社区的数目。于是必须有一种度量的方法,可以在计算过程中来衡量每个结果是不是相对最佳的结果。这同样也是算法好坏的评价指标。在GN算法中使用了模块度(modularity)这一概念。模块度的大小定义为社区内部的总边数和网络中总边数的比例减去一个期望值,该期望值是将网络设定为随机网络时同样的社区分配形成的社区内部的总边数和网络中总边数的比例的大小,模块度一般记为Q。在每次划分的时候计算Q值,当Q取最大值时则是此网络较为理想的划分。Q取值0~1,越大越好,实际中一般Q最高点在0.3~0.7。有时候,当不能或者不容易获取全部网络的数据时,可以用局部社区中的局部模块度来检测社区的合理性。局部模块度比全局模块度快很多,中小网络效果会比全局的差些,但是中等或大规模的网络中,局部模块度效果可能好要比全局的更好。其他的一些分图算法还包括:最大流最小割理论\cite{Jr2009Maximal}、谱二分的方法\cite{Pothen1990Partitioning}、Kernighan-Lin划分算法\cite{Kernigan1970An}和最小电导率分割算法【7】等等。

基于链接的社区发现算法其实也可以被看作是一种数据挖掘或者说机器学习聚类算法,相当于无监督的用户分类。因此这其中可以用到的无监督学习的相关技术包括:k-means算法、混合模型和层次聚类等等。

\subsection{基于内容的社区发现算法}
%\label{sec:requirements}
尽管基于链接的技术更加直观且基于社会学的同质原则,但也有两个原因导致它们在识别基于相似兴趣的用户社区方面存在缺陷。首先,许多社交关系不是基于用户的兴趣相似性,而是基于其他因素,如朋友和亲属关系,并不一定反映用户间的兴趣相似性。其次,许多有着相似兴趣的用户彼此之间可能并没有互相关注,以至于在网络之中似乎是没有关联的【8】。随着在线社交网络功能的不断增加,网络上除了用户之间的链接之外,还有许多用户自己提交的内容(称为社交内容)可用。用户可以维护个人资料页面,撰写评论,分享文章,标记照片和视频以及发布他们的状态更新。因此,研究人员已经探索了利用社交内容的主题相似性来检测社区的可能性。他们提出了基于内容或主题的社区检测方法,这样一来,不论社交网络结构如何,都可以检测到志同道合的社区用户【9】。

大多数基于内容的社区发现工作都侧重于检测社区文本内容的相似模型。比如,Abdelbary等人提出的算法【10】利用了高斯受限玻尔兹曼机(GRBM)来识别主题社区。尹志军等人【11】将社区发现与主题建模结合在一个统一的生成模型中,以检测在结构关系和潜在主题方面相互一致的用户群体。在他们的框架中,一个社区可以围绕多个主题形成,一个主题也可以在多个社区之间共享。Sachan等人【12】提出了概率方案,将用户的帖子、社交关系和交互类型结合起来发现Twitter中的潜在用户社区。在他们的工作中,他们考虑了三种类型的互动:传统推文、回复推文和转载推文。其他学者还提出了隐含狄利克雷分布模型(LDA)的变体来识别主题社区,例如作者-主题模型(Author-Topic model)【13】和社区-用户-主题模型(Community-User-Topic model)【14】。

另一个流派的工作将基于内容的社区发现问题建模为图聚类问题。这些方法都基于相似性度量标准,该度量标准能够根据用户都感兴趣的主题计算用户的相似度,并基于聚类算法来提取具有相似兴趣的用户群体(潜在社区)。例如,刘洪涛等人【15】提出了一种基于用户间主题距离(topic-distance)的聚类算法来检测社交标签网络中基于内容的社区。在这项工作中,LDA被用来提取标签中隐藏的主题。彭敦陆等人【16】提出了一个层次聚类算法来检测推文中的潜在社区。他们在新浪微博中使用了预定义的类别,并根据用户在每个类别中的兴趣程度计算了用户的配对相似度。

像基于链接的方法一样,基于内容的社区发现方法也可以转化为数据的聚类,这里的一个社区只是一组节点的集合。代表用户的节点与同一社区内的节点相似度较高,而与社区外的节点相似度较低。从这个意义上说,亲密关系确实是社区所需的凝聚力属性。

\subsection{融合链接和内容的社区发现算法}

基于内容的方法其实是为常规文本设计的,但是诸如Twitter或微博这类社交网络多是简短、混杂和非正式的社交内容。在这种情况下,社交内容本身并不是提取真实社区的可靠信息【17】。据推测,通过社交结构(即链接)丰富社交内容的确有助于我们找到更有意义的社区。研究人员们已经提出了几种方法将链接和内容信息结合起来用于社区发现。正如参考文献【18、19】中所述,它们可以拥有更好的性能。大多数这类方法是通过共享隐含变量这一手段来为社区成员制定链接和内容的综合生成模型。

Erosheva等人【20】介绍了Link-LDA,一种重叠的社区发现方法。它可以同时根据摘要(内容)和参考文献(链接)对科学类论文进行分类。在它们的生成模型中,论文被假定为摘要和参考文献的一对模型,每个部分都用LDA抽取特征。在摘要和参考文献中相似性都很高的文章倾向于有着相同的主题。与Link-LDA相反,Nallapti等人【21】没有将参考文献视作待处理的单词,并提出需要明确引用文本和参考文献之间的主题关系。他们提出了Pairwise-Link-LDA来模拟文档对之间的链接存在,并通过使用这些附加信息获得了更好的主题质量。其他利用LDA融合链接和内容的方法可以参考文献【22、23】。除了相似度生成模型外,还有其他一些方法将链接和内容信息结合起来用于社区发现,如谱聚类中利用矩阵分解和核聚变的方法【24、25】。

\subsection{重叠社区发现算法}

社区发现算法的常见方法是将网络划分为不相交的社区成员。这种方法忽略了个体可能属于两个或更多社区的可能性。但是,许多真实的社交网络都存在着社区的重叠【26】。例如,一个人可以属于多个社交群体,例如家庭群体和朋友群体。越来越多的研究人员开始探索允许社区重叠的新方法,即重叠社区(overlapping communities)。重叠社区引入了另一个变量,即不同社区中用户的成员身份,称为cover。由于与标准社区相比,重叠社区有大量可能的cover,因此检测此类社区代价就很高。

一些重叠的社区发现算法利用网络中用户的结构信息将网络的用户分成不同的社区。这类方法的主导算法是基于集团渗透理论(clique percolation theory)【27】。然而,LFM和OCG方法是基于对用户出入度适应函数的局部优化【28、29】。此外,一些模糊社区发现算法会计算每个节点属于每个社区的可能性,如SSDE和IBFO【30、31】。几乎所有的算法都需要先验信息来检测重叠的社区。例如,LFM需要一个参数来控制社区的大小。不过也有一些基于相似性的方法将社区看作分布在整个用户空间的隐含变量,如参考文献【32】。

然而,最近的研究集中在链接上。这最初是由Ahn等人【33】提出的,通过链接聚类(link clustering)找到的是链接的社区而不是用户的社区。言下之意是,虽然用户可以有许多不同的关系,但组内的关系在结构上是相似的。通过将链接划分为不重叠的组,每个用户可以通过继承其链接的社区而映射到多个社区。链接聚类方法显著加速了重叠社区的发现。

\section{论文主要工作}

通过查阅大量关于社交网络、社区发现、聚类等方面的文献资料,深入理解社交网络及重叠社区特性的基础上,认真研究社区发现相关算法,本文完成了如下工作:

(1)研究问题具体化。本文将社交网络转化为无向带权网络模型,针对无向网络找寻社区划分方案。在允许节点属于多个社区的情况下,使得划分后的社区内节点关系紧密,社区间节点关系疏远。

(2)CDMMLPA 算法的设计。标签传播算法主要分为标签分配和传播2个阶段。CDMMLPA算法中并未为每个节点分配标签,而是以小社区为单位,为每个小社区分配唯一的标签。采用模块度最大化的方法进行标签传播,由于模块度最大化是一个单向操作,因此能获得高质量的社区。CDMMLPA算法中充分考虑了社区结构的属性,以确保得到稳定的社区。

\section{论文组织结构}

本论文主要对重叠社区的挖掘算法进行研究,并设计了相应的优化算法。本文主要包括四大章节,其主要的结构组织如下:

第一章为绪论。主要介绍了课题的背景、意义、国内外现状以及本课题的主要研究内容。其中,重点介绍了各类社区发现算法的国内外研究现状。

第二章为相关工作。首先介绍了复杂网络的定义;然后介绍了复杂网络中社交网络的特性;接下来给出了社区发现的定义,从社区结构定义和社交网络网络模型着手;最后介绍了社区结构的评价指标。

第三章为基于模块度最大化的标签传播社区发现算法。主要介绍本文设计的一种基于模块度最大化的标签传播社区发现算法(CDMMLPA)。本章首先是针对算法中的一些特殊名词给出了相对应的定义;然后详细解释了算法实现过程;最后给出了算法的伪代码。

第四章为社区发现算法相关实验与评估。首先介绍了实验用的数据集;然后是对比实验部分,主要和经典标签传播算法LPA和改进的标签传播算法LPAm+的比较;比较维度主要是模块度、强社区数目和运行时间;最后是实验总结。
