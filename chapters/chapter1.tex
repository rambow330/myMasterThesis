%%==================================================
%% chapter01.tex for BIT Master Thesis
%% modified by 朱杰
%%==================================================
\chapter{绪论}
% 绪论的撰写思路是:
% 背景:社交软件的普及,社交由线下转线上,社交网络数据的爆炸增长,分析海量社交网络数据很有意义,针对大规模数据,传统社区发现算法不再适用。
% 意义:针对社交网络的社区发现的意义,有利于商业推荐,舆情控制等等;有利于社交网络数据的分布式存储,降低通信开销;社区发现算法不止用于社交网络,还有别的很多领域都有应用。
\section{研究背景和意义}
近年来,随着科技的发展和网络的不断普及,在线社交网络如今已经成为互联网时代最为基础的一部分。诸如微信、微博、Facebook、Twitter和GitHub等等国内外的社交类平台软件的出现,使得人们可以更加有高效的沟通交流。在如今的移动互联网时代下,人们的社交重心由线下更多的转到了线上。线上的社交也确实带来了很多的便捷。人们不再有地域的限制,可以轻松的与亲朋好友时刻保持联系;人们通过社交平台可以迅速的认识了解一个人并与之成为朋友;人们可以扮演起在日常生活中无法扮演的角色,任何人都可以成为信息的分享者和传播者。

在种类繁多的在线社交软件中,人们的参与度越来越高。微博发布的2018年第一季度的财报显示,2018年3月的月活跃用户数(MAUs)较上年同期净增约7000万,达到4.11亿,其中月活跃用户数中93\%为移动端用户;2018年1月15日在广州举行的微信公开课上,微信创始人、腾讯高级副总裁张小龙指出微信用户量已超10亿;而国外的Facebook更是早在2017年就已超20亿用户。社交网络具有传播迅速、传播广泛、自发性和言论相对自由等特点。面对这么巨大的用户量,不仅仅是普通个人用户,我们可以发现各类官方媒体也都已入驻这些知名在线平台,以达到更快捷、更方便和更广泛的宣传目的。

面对着如此巨大的用户量,这些成熟的社交软件每天都会产生海量的用户数据。在这个大数据的时代,这些看似杂乱无章、毫无交集的数据中,其实蕴含了丰富的信息等待着人们去挖掘与分析。在这样的背景下,如果将社交平台中所有用户抽象成点,而用户与用户之间的关联抽象成边,这就抽象出了一张网络关系图,那么对于在互联网这个虚拟世界中形成的这样一张巨大而又复杂的社交网络的研究与分析就显得意义非凡。而面对如今的大数据环境,传统的一些研究分析手段都已经无法再胜任。因此,为了更好的利用社交网络给人们带来的便捷,同时又能避免产生危害,就产生了社交网络分析(Social Network Analysis)这一研究领域。它是一门横跨信息学、数学、计算机技术、社会学、管理学和心理学等学科的交叉科学,主要研究的是社交网络的网络结构和其演化、社交网络中的群体及其互动、社交网络中的信息及其传播。

正所谓物以类聚,人以群分。社交网络中亦是如此。网络图内部连接比较紧密的节点子集合对应的子图被称之为社区(Community)。网络图中包含一个个社区的现象称之为社区结构(Community Structure)。社区结构是社交网络结构的一个普遍特征。而给定一个网络图,找出其社区结构的过程就叫做社区发现(Community Detection),也有学者将其翻译为社区检测。社区发现正是社交网络分析的一种手段。挖掘社交网络中的社区在人物分析、商业个性化推荐等领域有着很关键的作用。为了获得更大的用户群体以获取更多的流量关注或者刺激用户更多的消费,在线求职招聘类平台要为用户推荐适合用户需求的职位,在线购物消费类平台要为用户推荐符合用户需求的商品,在线社交网络平台要为用户推荐和用户兴趣相投或者相关的好友,在线新闻媒体平台要为用户推荐符合用户口味的相关讯息。在这样的背景下,一个优秀的个性化推荐系统就显得尤为重要。而在推荐系统中,免不了要对用户群体进行分类,而社区发现所形成的社区在此就有着先天的优势。简而言之,对社交网络的社区发现无非也就是对社交网络中的用户进行分类(或者说聚类)。在同一个社区内的用户群体,往往有着相似的兴趣爱好。因此如果对整个用户群体进行社区发现,那么比如说在为用户进行商品个性化推荐时,可以重点关注与该用户在同一个社区的其他用户购买过而他未购买过的商品,这样用户的接受率也会高很多。当然,推荐系统是一个复杂的AI系统,利用了远不止社区发现这一种手段。

随着用户量的越发增加,在线社交媒介的广泛使用带来的问题也越来越多。这也对社交网络的规范化和整治不断提出挑战。在人们享受社交网络带来的乐趣和便利之时,同样也有不法分子为了金钱或其他利益利用社交网络缺乏规范又利于传播等特点进行违法犯罪,包括诈骗、散布暴力恐怖信息或谣言等等。最近国家广电总局也整治了一大批社交媒体,封杀了一系列严重违规的软件,即使是今日头条、抖音、快手这样的大公司也面临着很大的危机,被勒令整改。对社交网络的社区发现研究同样可以对舆情监控这一领域做出相应的贡献,在锁定犯罪团伙之间的网络关系、挖掘舆情传播起源点、控制舆论导向等方面社区发现算法均有一定的帮助。

此外,对大规模社交网络进行社区检测在社交网络用户数据的分布式存储上也有很大的意义。现在的社交网络数据量巨大,往往都是以图结构的形式存储于不同台服务器。在图结构各个节点之间的通信开销之中,跨服务器的节点通信开销是主要的,而服务器内部通信往往要不了多少花销。因此,如果把隶属于同一社区的用户数据都存储于同一台服务器中,就可以很大程度上的降低服务器之间的通信开销,也就间接减少了各类图计算算法的运行时间。

对社区发现算法的研究其实远不止社交网络这一领域,社区发现也是对复杂网络的一种分析手段。除了社交网络,还有科学文献引用网络、生物学分子结构分析等等领域也都使用着社区发现算法。多年来众多学者的贡献使得对社区发现算法的研究已经是比较成熟了,但是依然存在着问题。现如今的社交网络数据往往是海量的,对如此大规模数据的社交网络进行社区发现时,过往的一些经典算法显然已经无法胜任。因此,在这样的背景下,提出一种适应于当前大规模社交网络的社区发现算法就显得尤为迫切了。

% 国内外研究现状的撰写思路是:
% 要强调现状(近三年的论文一定要有)
% 分为了三大类算法对现状进行描述,主要是就是说清楚现有算法的特性以及存在的问题,方便下一节论文主要工作中引出本人提出的算法。
% 也可列出别的领域可以应用于该领域的技术
\section{国内外研究现状}
社区发现算法根据不同的角度可以有很多种分类,本文将现有的社区发现算法分为三大类:基于链接的方法,基于内容的方法和融合链接与内容的方法。基于链接的方法也就是基于网络拓扑结构的方法。它将社交网络看做一张网络图,用户为节点,关系为边。基于内容的方法主要是基于社交网络中用户的个人信息和发表过的内容来进行社区发现。基于内容的方法也可被称为基于主题的方法,或者基于节点相似度的方法。而融合的方法则同时关注网络拓扑结构和用户属性,以此来获取更高质量的社区。

% \subsection{基于链接的社区发现算法}
McPherson等人最早提出了同质性原则:相似性产生联系\cite{Mcpherson2001Birds}。这也被认为是大部分社区定义的最早的参考准则,同时这也有利于发现社区。人们总是习惯于和自己相似的人结交关系。不论是友谊还是工作关系,任何关系网络中人们的缔结方式均有此趋势。因此,在社交网络中两个用户之间所谓的链接关系被认为是一个可以证明彼此之间是有着某些共同点的证据。

在基于链接的社区发现算法中,社交网络是由一张图为模型,节点代表社区成员,边代表成员之间的关系或者交互。这里社区所需的凝聚力属性就是成员之间的链接。在社区之中链接是较为密集的,而在社区之间链接相对而言较为稀疏。文献\cite{Fortunato2009Community}分别将原始图结构中的组件和派系当做是已知的社区。然而,更多的更有意义的社区可以通过基于图划分(聚类)的方法来检测得到,其目标就是尽可能减少社区之间边的数量,并且使得同一个社区中的节点之间就可以有更多的内连接,而与别的社区中节点的外连接就可以减少。大部分的方法都是基于二分迭代:不断地将一个社团划分为两个社团。然而在复杂网络中社区的数量显然是无法预先得知的,在这个层面上,GN算法\cite{2002Community}是最为广泛使用的基于链接的社区发现算法。GN算法的基本思想是删除社区之间的连接,使得剩下的每个连通部分就是一个社区。作者巧妙地借助了最短路径这一思想。GN算法中定义一条边的边介数(Betweeness)为网络中所有节点之间的最短路径中通过这条边的数量,而边介数高的边要比介数低的边更可能是社区之间的边。因为两个社区中的节点之间的最短路径都要经过那些社区之间的边,所以它们的边介数会很高。

然而,GN算法的一个缺点就是无法评价找出的社区发现结果是否最优。为了解决这一问题,就必须有一种度量的方法,可以在计算过程中来衡量每个结果是不是相对最佳的结果。这同样也是算法好坏的评价指标。Newman等人提出了模块度(Modularity)\cite{NewmanFind}这一概念。模块度的大小定义为社区内部的总边数和网络中总边数的比例减去一个期望值,该期望值是将网络设定为随机网络时同样的社区分配形成的社区内部的总边数和网络中总边数的比例的大小,模块度一般记为Q。在每次划分的时候计算Q值,当Q取最大值时则是此网络较为理想的划分。Q取值,越大越好,实际中一般Q最高点在0.3至0.7。有时候,当不能或者不容易获取全部网络的数据时,可以用局部社区中的局部模块度来检测社区的合理性。局部模块度比全局模块度快很多,中小网络效果会比全局的差些,但是中等或大规模的网络中,局部模块度效果可能好要比全局的更好。GN算法还有一个缺点就是时间复杂度过高,只适用于规模较小的网络。对于数据量惊人的社交网络,GN算法约$O(N^3)$的时间复杂度明显无法胜任对其社区发现的工作。

考虑到社交网络的数据集过于庞大,要想适应于大规模的社交网络,算法的速度就显得至关重要。而这就必须要提到Raghavan等人提出的LPA算法\cite{Raghavan2007Near}了,这是迄今为止速度最快的社区发现算法,仅仅只有线性时间的复杂度,简单易理解且不需要任何先验信息。但是该算法由于标签更新阶段是随机顺序,并没有考虑不同节点的重要性,容易导致结果产生震荡,实验结果通常不稳定。研究人员围绕LPA算法不断进行改进,在算法的传播规则、收敛条件和更新策略等方面进行相关研究。例如朱福喜等人提出了一种基于标签传播概率的算法\cite{Liu2016A}。

基于链接的社区发现算法其实也可以被看作是一种数据挖掘或者说机器学习聚类算法,相当于无监督的用户分类。因此这其中可以用到的无监督学习的相关技术包括:k-means算法、混合模型和层次聚类等等。此外,其他的一些可应用于社区发现的分图算法还包括:最大流最小割理论\cite{Jr2009Maximal}、谱二分的方法\cite{Pothen1990Partitioning}、Kernighan-Lin划分算法\cite{Kernigan1970An}和最小电导率分割算法\cite{Leskovec2005Graphs}等等。

%\subsection{基于内容的社区发现算法}
尽管基于链接的技术更加直观且基于社会学的同质原则,但也有两个原因导致它们在识别基于相似兴趣的用户社区方面存在缺陷。首先,许多社交关系不是基于用户的兴趣相似性,而是基于其他因素,如朋友和亲属关系,并不一定反映用户间的兴趣相似性。其次,许多有着相似兴趣的用户彼此之间可能并没有互相关注,以至于在网络之中似乎是没有关联的\cite{Deng2013Interaction}。随着在线社交网络功能的不断增加,网络上除了用户之间的链接之外,还有许多用户自己提交的内容(称为社交内容)可用。用户可以维护个人资料页面,撰写评论,分享文章,标记照片和视频以及发布他们的状态更新。因此,研究人员已经探索了利用社交内容的主题相似性来检测社区的可能性。他们提出了基于内容或主题的社区检测方法,这样一来,不论社交网络结构如何,都可以检测到志同道合的社区用户\cite{Natarajan2013Community}。

大多数基于内容的社区发现工作都侧重于检测社区文本内容的相似模型。比如,Abdelbary等人提出的算法\cite{Abdelbary2014Utilizing}利用了高斯受限玻尔兹曼机(GRBM)来识别主题社区。尹志军等人\cite{Yin2012Latent}将社区发现与主题建模结合在一个统一的生成模型中,以检测在结构关系和潜在主题方面相互一致的用户群体。在他们的框架中,一个社区可以围绕多个主题形成,一个主题也可以在多个社区之间共享。Sachan等人\cite{Sachan2012Using}提出了概率方案,将用户的帖子、社交关系和交互类型结合起来发现Twitter中的潜在用户社区。在他们的工作中,他们考虑了三种类型的互动:传统推文、回复推文和转载推文。其他学者还提出了隐含狄利克雷分布模型(LDA)的变体来识别主题社区,例如作者-主题模型(Author-Topic model)\cite{RosenZvi2012The}和社区-用户-主题模型(Community-User-Topic model)\cite{Zhou2006Probabilistic}。

另一个流派的工作将基于内容的社区发现问题建模为图聚类问题。这些方法都基于相似性度量标准,该度量标准能够根据用户都感兴趣的主题计算用户的相似度,并基于聚类算法来提取具有相似兴趣的用户群体(潜在社区)。例如,刘洪涛等人\cite{Liu2014Community}提出了一种基于用户间主题距离(Topic-Distance)的聚类算法来检测社交标签网络中基于内容的社区。在这项工作中,LDA被用来提取标签中隐藏的主题。彭敦陆等人\cite{Peng2015DICH}提出了一个层次聚类算法来检测推文中的潜在社区。他们在新浪微博中使用了预定义的类别,并根据用户在每个类别中的兴趣程度计算了用户的配对相似度。

像基于链接的方法一样,基于内容的社区发现方法也可以转化为数据的聚类,这里的一个社区只是一组节点的集合。代表用户的节点与同一社区内的节点相似度较高,而与社区外的节点相似度较低。从这个意义上说,亲密关系确实是社区所需的凝聚力属性。

%\subsection{融合链接和内容的社区发现算法}
基于内容的方法其实是为常规文本设计的,但是诸如Twitter或微博这类社交网络多是简短、混杂和非正式的社交内容。在这种情况下,社交内容本身并不是提取真实社区的可靠信息\cite{Yang2009Combining}。通过社交结构(即链接)丰富社交内容有助于我们找到更有意义的社区。研究人员们已经提出了几种方法将链接和内容信息结合起来用于社区发现。正如参考文献\cite{Cohn2001The,Getoor2003Learning}中所述,它们可以拥有更好的性能。大多数这类方法是通过共享隐含变量这一手段来为社区成员制定链接和内容的综合生成模型。

%subsection{重叠社区发现算法}
社区发现算法的常见方法是将网络划分为不相交的社区成员,而这种方法忽略了个体可能属于两个或更多社区的可能性。但是,许多真实的社交网络都存在着社区的重叠\cite{Xie2013Overlapping}。例如,一个人可以属于多个社交群体,例如家庭群体和朋友群体。越来越多的研究人员开始探索允许社区重叠的新方法,即重叠社区(Overlapping Communities)。重叠社区引入了另一个变量,即不同社区中用户的成员身份,称为cover。由于与标准社区相比,重叠社区有大量可能的cover,因此检测此类社区代价就很高。

一些重叠的社区发现算法利用网络中用户的结构信息将网络的用户分成不同的社区。这类方法的主导算法是基于集团渗透理论(Clique Percolation Theory)\cite{Palla2005Uncovering}。然而,LFM和OCG方法是基于对用户出入度适应函数的局部优化\cite{Lancichinetti2012Detecting,Becker2012Multifunctional}。此外,一些模糊社区发现算法会计算每个节点属于每个社区的可能性,如SSDE和IBFO\cite{Magdon2010SSDE,Lei2013Clustering}。几乎所有的算法都需要先验信息来检测重叠的社区。例如,LFM需要一个参数来控制社区的大小。不过也有一些基于相似性的方法将社区看作分布在整个用户空间的隐含变量,如参考文献\cite{Ren2007A}。

Erosheva等人\cite{Erosheva2004Mixed}介绍了Link-LDA,一种重叠的社区发现方法。它可以同时根据摘要(内容)和参考文献(链接)对科学类论文进行分类。在它们的生成模型中,论文被假定为摘要和参考文献的一对模型,每个部分都用LDA抽取特征。在摘要和参考文献中相似性都很高的文章倾向于有着相同的主题。与Link-LDA相反,Nallapti等人\cite{Nallapati2008Joint}没有将参考文献视作待处理的单词,并提出需要明确引用文本和参考文献之间的主题关系。他们提出了Pairwise-Link-LDA来模拟文档对之间的链接存在,并通过使用这些附加信息获得了更好的主题质量。其他利用LDA融合链接和内容的方法可以参考文献\cite{Dietz2007Unsupervised,Gruber2008Latent}。除了相似度生成模型外,还有其他一些方法将链接和内容信息结合起来用于社区发现,如谱聚类中利用矩阵分解和核聚变的方法\cite{Zhu2007Combining,Yu2008Clustering}。

% 论文主要工作的撰写思路:
% 首先要有个大致的引出工作的导语,然后说出自己的算法就好了?
\section{论文主要工作}

通过查阅大量关于社交网络、社区发现、聚类等方面的文献资料,深入理解社交网络及重叠社区特性的基础上,认真研究社区发现相关算法,本文完成了如下工作:

(1)CDABSLP 算法的设计。
该算法从两个方面改善 LPA 算法不稳定的问题:首先,计算
网络中每个节点的节点影响值并按节点影响值降序排列作为节点标签更新的顺
序,取代传统 LPA 算法中节点更新顺序随机确定的方法;其次,在每次标签更
新迭代过程中,当传统的标签计算方法返回多个标签时,提出一种新的标签计算
公式,计算返回标签的影响强度,在返回的多个标签中重新选择一个影响强度最
大的标签作为该节点的新标签,以此替代传统 LPA 算法中随机选择一个标签的
方法。CDABSLP算法既保持了传统 LPA 算法的优点,还解决了 LPA 算法不稳定
的问题,该算法能够得到稳定的社区结构。大量实验结果表明CDABSLP算法的性
能优于目前一些代表性的社区发现算法。 

(2)OCDABSLP 算法的设计。该算法OCDABSLP 算法采用同步更新策略,在标签更新过程中,当一个节点拥有
的所有标签对应的隶属度都小于 $1/v$,且此时有多个标签的隶属度同时取最大值
时,将节点影响值引入到标签隶属度计算公式中,得到这些标签的影响强度,保
留影响强度最大的标签,取代传统 COPRA 算法随机保留其中一个标签的方法,
提高算法的稳定性。在重叠 LFR 数据集上的实验结果表明 OCDABSLP 算法解决
了 COPRA 算法不稳定的问题,能够检测得到较优的重叠社区结构,验证了本章
提出的稳定策略在重叠社区发现算法 COPRA 算法中的适用性。



\section{论文组织结构}

本论文主要对重叠社区的挖掘算法进行研究,并设计了相应的优化算法。本文主要包括四大章节,其主要的结构组织如下:

第一章为绪论。主要介绍了课题的背景、意义、国内外现状以及本课题的主要研究内容。其中,重点介绍了各类社区发现算法的国内外研究现状。

第二章为相关工作。首先介绍了复杂网络中社交网络的相关概念,包括社交网络概述、社交网络的统计特性以及社交网络的典型特征;接下来从不同角度给出了社区的定义,从而引出了社区发现这一概念。

第三章为基于稳定标签传播的非重叠社区发现算法。主要介绍本文设计的一种基于节点影响度的稳定的标签传播非重叠社区发现算法(CDABSLP)。本章首先是对经典标签传播算法做了简单介绍;然后详细介绍CDABSLP 算法的设计思路、核心思想和关键步骤等,接着在真实网络以及人工基准网络上的实验,并与其他基准算法进行对比实验,以此来分析算法的效果,最后进行实验总结。

第四章为基于稳定标签传播的重叠社区发现算法。主要介绍本文设计的一种基于节点影响度的稳定的标签传播重叠社区发现算法(OCDABSLP)。本章先简单介绍了多标签传播算法(COPRA),然后详细介绍OCDABSLP 算法的设计思路、核心思想和关键步骤等,接着在真实网络以及人工基准网络上的实验,并与其他基准算法进行对比实验,以此来分析算法的效果,最后进行实验总结。