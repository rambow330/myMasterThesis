%%==================================================
%% conclusion.tex for BIT Master Thesis
%% modified by 朱杰
%%==================================================


\begin{conclusion}

% 本文采用……。{\color{blue}(结论作为学位论文正文的最后部分单独排写,但不加章号。结论是对整个论文主要结果的总结。在结论中应明确指出本研究的创新点,对其应用前景和社会、经济价值等加以预测和评价,并指出今后进一步在本研究方向进行研究工作的展望与设想。结论部分的撰写应简明扼要,突出创新性。)}

% 在查阅了大量社交网络、社区发现、图划分以及图聚类等相关资料之后,详尽地了解了社交网络的统计特性及其典型特征,仔细地对普通社区发现算法以及重叠社区发现算法进行了研究。社交网络具有“小世界”和“无标度”的特征,本文决定对简单、快速和无需任何先验知识的标签传播算法(LPA)的基础上进行改进,解决标签传播算法不稳定的问题。同样针对重叠社区发现问题,对多标签传播算法(COPRA)进行改进,在保证稳定性的基础上尽可能的保持算法高效执行效率。下面是本文提出的两种算法的概述。

% (1)CDABSLP算法的设计。本文提出了一种基于稳定标签传播的非重叠社区发现算法,简称CDABSLP。该算法在标签更新的时候采用异步更新策略;然后引入了一种综合节点自身重要性以及邻居节点重要性的节点影响力模型,按照节点影响力大小降序排列的顺序更新标签;在标签选择方式上,在节点影响力的基础上引申出了标签影响力模型,选择标签影响力最大的标签对原始标签进行更新;最终根据每个节点对应的标签进行社区划分。实验结果表明,CDABSLP算法具有较好的稳定性、较高的运行效率和良好的社区划分质量。

% (2)OCDABSLP算法的设计。本文提出了一种基于稳定标签传播的重叠社区发现算法,简称OCDABSLP。该算法在标签更新的时候采用同步更新策略;在标签选择方式上,对于邻居节点标签隶属度均小于既定阈值,且含有多个不同的最大隶属度标签的情况,选择其中标签影响力最大的进行更新;最终同样根据每个节点对应的标签进行社区划分,含有多个标签的节点即对应着存在于多个社区的重叠节点。实验结果表明,OCDABSLP算法具有较好的稳定性、较高的运行效率和良好的重叠社区划分质量。

目前,社区发现已成为社交网络研究领域中的热点问题之一,本课题以社交网络数据为研究对象,以发现网络中的社区结构为研究目标,基于标签传播思想,提出了两种社区发现算法。在进行社区发现的过程中,充分考虑了节点自身的重要性、邻居节点的重要性、节点与社区的关系和社区与社区的关系等多方面信息。下面是本文的主要成果:
% 在查阅了大量社交网络、社区发现、图的划分以及图的聚类等相关资料之后,考虑到社交网络大规模以及“小世界”、“无标度”的特征,本文在简单快速的标签传播算法(LPA)的基础上进行改进,解决标签传播算法不稳定的问题。同样在重叠社区发现上,对多标签传播算法(COPRA)进行改进。下面是本文的主要成果:
\begin{enumerate}
    \item 为了得到稳定的非重叠社区发现结果,提出了一种基于稳定标签传播的非重叠社区发现算法,简称CDABSLP。该算法在标签更新的时候采用异步更新策略;然后提出了一种综合节点自身重要性以及邻居节点重要性的节点影响力模型,按照节点影响力大小降序排列的顺序更新标签;在标签选择方式上,在节点影响力的基础上提出了标签影响力模型,选择标签影响力最大的标签对原始标签进行更新;最终根据每个节点对应的标签进行社区划分。实验结果表明,CDABSLP算法具有较好的稳定性、较高的运行效率和良好的社区划分质量。
    \item 为了得到稳定的重叠社区发现结果,提出了一种基于稳定标签传播的重叠社区发现算法,简称OCDABSLP。该算法在标签更新的时候采用同步更新策略;在标签选择方式上,对于邻居节点标签隶属度均小于既定阈值,且含有多个不同的最大隶属度标签的情况,选择其中标签影响力最大的进行更新;最终同样根据每个节点对应的标签进行社区划分,含有多个标签的节点即对应着存在于多个社区的重叠节点。实验结果表明,OCDABSLP算法具有较好的稳定性、较高的运行效率和良好的重叠社区划分质量。
\end{enumerate}

本文的研究仅在社交网络的社区发现方法的几个角度取得了一些阶段性的研究成果,在未来的研究工作中还可以在以下几个方面进一步扩展:

\begin{enumerate}
    \item 随着时间的流逝,社交网络的内容也是在不断的发生变化,节点和边可能增加,也可能减少。在某一时刻的社交网络中发现的社区结构,很可能在另一时刻产生很大变化。如何在变化的社交网络上进行动态社区发现,并从中挖掘出社区变化甚至整个网络变化的规律和特点,可能成为未来研究的方向之一;
    \item 现在在现实世界中的社交网络是巨大的,而在超大规模的社交网络上如何实现快速的社区发现,降低时间复杂度的同时,又能取得较好的社区发现结果,需要在未来进一步深入研究;
    % \item 随着相关领域研究的不断深入,网络中可能会出现更多新的属性特征。
    % 如何发现不同领域新属性特征并融合应用于复杂网络的社区发现,从而更全面
    % 地获得符合真实情况的社区结构,是未来需要考虑的内容之一;
    \item 随着社区发现技术的不断提升,社区发现结果将更加精确高效。但如何发现社区仅是研究的基础内容之一,社区发现技术与实际应用领域的结合还不够广泛深入。目前,社区发现仅与话题检测、用户推荐等领域进行了融合。如何将社区发现技术及其挖掘出的社区信息内容应用到更多实际领域中,是未来需要探索开拓的方向之一。
\end{enumerate}

\end{conclusion}