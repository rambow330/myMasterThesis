%%==================================================
%% conclusion.tex for BIT Master Thesis
%% modified by 朱杰
%%==================================================


\begin{conclusion}

% 本文采用……。{\color{blue}(结论作为学位论文正文的最后部分单独排写,但不加章号。结论是对整个论文主要结果的总结。在结论中应明确指出本研究的创新点,对其应用前景和社会、经济价值等加以预测和评价,并指出今后进一步在本研究方向进行研究工作的展望与设想。结论部分的撰写应简明扼要,突出创新性。)}

在查阅了大量社交网络、社区发现、图的划分以及图的聚类等相关资料之后,考虑到社交网络大规模以及“小世界”、“无标度”的特征,本文在简单快速的标签传播算法(LPA)的基础上进行改进,解决标签传播算法不稳定的问题。同样在重叠社区发现上,对多标签传播算法(COPRA)进行改进。下面是本文的主要成果:
\begin{enumerate}
    \item 为了得到稳定的非重叠社区发现结果,提出了一种基于稳定标签传播的非重叠社区发现算法,简称CDABSLP。该算法在标签更新的时候采用异步更新的策略;然后提出了一种综合节点自身重要性以及邻居节点重要性的节点影响力模型,将节点更新顺序设为节点影响力的降序;最后在标签选择方式上,提出了一种标签影响力模型。通过验证实验证明CDABSLP算法在具有良好的稳定性的同时兼具较高的运行效率,在多个评价指标之上社区发现准确率均有不错的表现;
    \item 为了得到稳定的非重叠社区发现结果,提出了一种基于稳定标签传播的重叠社区发现算法,简称OCDABSLP。在简单分析了原始COPRA算法存在的缺陷之后,针对相应的缺陷介绍了稳定性解决方案。OCDABSLP算法采用了同步更新标签的策略,这点与CDABSLP算法不同;在标签选择方式上,对于邻居节点标签隶属度均小于既定阈值,且含有多个不同的最大隶属度标签的时候,选择最大隶属度标签中标签影响力模型计算值最大的那一个进行更新。在算法最终收敛之后,根据每个节点含有的标签将其划分到相应的社区,含有多个标签的节点即对应着存在于多个社区的重叠节点。通过真实网络以及人工网络中的验证实验,可知CDABSLP算法在具有良好的稳定性的同时兼具较高的运行效率,在多个评价指标之上社区发现准确率均有不错的表现。
\end{enumerate}

% 本文的研究仅为在社交网络中社区发现方法的几个角度取得了一些阶段性
% 的研究成果,在未来的研究工作中还可以在以下几个方面进行进一步扩展:

% \begin{enumerate}
%     \item 随着时间的流逝,复杂网络的内容也是在不断的发生变化,节点和边可
%     能增加,也可能减少。在某一时刻的复杂网络中发现的社区结构,很可能在另
%     一时刻产生很大变化。如何在变化的复杂网络上进行动态社区发现,并从中挖
%     掘出社区变化甚至整个网络变化的规律和特点,可能成为未来研究的方向之一;
%     \item 现在在现实世界中的社交网络是巨大的,而在超大规模的社交网络上如何实现快速
%     的社区发现,降低时间复杂度的同时,又能取得较好的社区发现结果,需要在
%     未来进一步深入研究;
%     \item 随着相关领域研究的不断深入,网络中可能会出现更多新的属性特征。
%     如何发现不同领域新属性特征并融合应用于复杂网络的社区发现,从而更全面
%     地获得符合真实情况的社区结构,是未来需要考虑的内容之一;
%     \item 随着社区发现技术的不断提升,社区发现结果将更加精确高效。但如何
%     发现社区仅是研究的基础内容之一,社区发现技术与实际应用领域的结合还不
%     够广泛深入。目前,社区发现仅与话题检测、用户推荐等领域进行了融合。如
%     何将社区发现技术及其挖掘出的社区信息内容应用到更多实际领域中,是未来
%     需要探索开拓的方向之一。
% \end{enumerate}

\end{conclusion}